\chapter{JPEG 图像有损压缩}
JPEG(Joint Photographic Experts Group,联合图像专家组),是一种常用的图像存储格式,是一种高效率的压缩格式,文件格式是JPEG(联合图像专家组)标准的产物,该标准由ISO与CCI TT(国际电报电话咨询委员会)共同制定,是面向连续色调静止图像的一种压缩标准。

JPEG有损图像压缩主要包括图像分块、颜色空间转换、色度采样、离散余弦变换DCT、量化、编码等步骤。

\section{图像分块}
在JPEG流程中,对整张图像进行一步压缩不太现实,首先是算法的复杂度较大需要大量消耗计算资源,其次,如果图像较大,甚至无法进行处理,因此原始图像会被分割成大小为$8\times 8$的图像子块,在后续的压缩编码处理中,会以图像子块为单位进行处理。

例如图\ref{Fig.patch}中的160\times160大小的原始图像,就可以分成20\times20个8\times8图像块。
\begin{figure}[!htb]
    \centering
    \includegraphics[width=0.8\textwidth]{pages/jpeg/patch.jpg}
    \caption{图像分块}
    \label{Fig.patch}
\end{figure}

\section{颜色空间转换与色度采样}
\subsection{颜色空间转换与色度采样的理论基础}

在人眼视觉系统(Human Visual System, HVS)理论\cite{HVS}中,对光线的感知主要依靠视网膜上的细胞,在视网膜上两种感光细胞,一种叫做视锥细胞,另—种叫做视杆细胞。一只眼睛里面大约分别有七百万视锥细胞和一亿两千万视杆细胞。

视锥细胞(cone cell)负责感知较强的光照和色彩,视网膜中有三种视锥细胞,它们有重叠的频率响应曲线,但响应强度有所不同,分别对红(570nm), 绿(535nm), 蓝(445nm)光有最敏感,共同决定了人眼对色彩的感觉。

视杆细胞(rod cell)只能感知光照而不能感知色彩,在视网膜中只有一种视杆细胞,因此没有颜色感觉,但对光的灵敏度非常高,可以看到非常暗的物体,这也就是人眼在暗光下无法分辨颜色的原因。

由于主要感知光照视杆细胞的数量远远多于主要感知色彩的视锥细胞,因此人眼对光照的变化会比对颜色的变化更加敏感,图像有损压缩的主要根据此特性来对颜色(色度)进行进一步采样来压缩图像以减少图像数据中的冗余部分,该冗余指的是无法被人眼所感知的部分。

根据人眼的这一特点,在图像和视频编码中,保留色度信息,大幅度对色度进行采样的技术有着十分广泛的应用,尤其是在视频编码中通过减少色度信息保留亮度信息以降低视频带宽。同样的,JPEG标准中也根据此特点对色度进行采样。

\subsection{颜色空间转换}
根据上述理论基础,在JPEG标准中,首先将 RGB 颜色空间转化为 YUV 颜色空间(也叫 YCbCr)其中,Y 是亮度 (Luminance)(也叫亮度信息),U 和 V 表示色度 (Chrominance)(也叫色差信息)。

研究表明,红绿蓝三基色所贡献的亮度不同,绿色所贡献亮度最多,蓝色所贡献亮度最少,这是由于人眼对相同强度不同波长的光具有不同的敏感度,其中对绿色光(550nm)产生最大的光强敏感度,而对蓝色光(445nm)产生光强敏感度较小。假定红色贡献为 $K_R$,蓝色贡献为 $K_B$,则亮度可以表示为

\begin{equation}
    Y = K_R \cdot R + (1-K_R-K_B) \cdot G + K_B \cdot B
    \label{Eq.yuv}
\end{equation}

根据经验值 $K_R=0.299, K_B=0.114$,则有
\begin{equation}
    Y = 0.299 \cdot R + 0.587 \cdot G + 0.114 \cdot B
\end{equation}

蓝色和红色的色差为
\begin{equation}
    \begin{aligned}
        Y   &= 0.299   \cdot R + 0.587    \cdot G + 0.114 \cdot B \\
        C_b &= -0.1687 \cdot R - 0.3313   \cdot G + 0.5 \cdot B +128\\
        C_r &= 0.5   \cdot R - 0.4187   \cdot G - 0.0813 \cdot B +128\\
    \end{aligned}    
\end{equation}

矩阵表示为
\begin{equation}
    \begin{bmatrix}
        Y \\ C_b \\ C_r
    \end{bmatrix}
    =\begin{bmatrix}
        0.299   & 0.587    & 0.114 & \\
        -0.1687 & -0.3313   & 0.5 & \\
        0.5     & -0.4187   & -0.0813 &
    \end{bmatrix}
    \begin{bmatrix}
        R \\ G \\ B
    \end{bmatrix}
    +\begin{bmatrix}
        0 \\ 128 \\ 128
    \end{bmatrix}
\end{equation}


\begin{figure}[ht]
    \centering
    \begin{subfigure}{0.245\textwidth}
      \includegraphics[width=\linewidth]{pages/jpeg/origin.jpg}
      \caption{RGB图像}
      \label{Fig:RGB2YUV-o}
    \end{subfigure}
    \begin{subfigure}{0.245\textwidth}
      \includegraphics[width=\linewidth]{pages/jpeg/yuv-y.jpg}
      \caption{YUV-Y通道}
      \label{Fig:RGB2YUV-y}
    \end{subfigure}
    \begin{subfigure}{0.245\textwidth}
      \includegraphics[width=\linewidth]{pages/jpeg/yuv-u.jpg}
      \caption{YUV-U通道}
      \label{Fig:RGB2YUV-u}
    \end{subfigure}
    \begin{subfigure}{0.245\textwidth}
      \includegraphics[width=\linewidth]{pages/jpeg/yuv-v.jpg}
      \caption{YUV-V通道}
      \label{Fig:RGB2YUV-v}
    \end{subfigure}
    \caption{RGB图像转化到YUV色彩通道}
    \label{Fig:RGB2YUV}
\end{figure}


\subsection{色度采样}
将 RGB 颜色空间转化为 YUV 颜色空间后,需要对色度进行二次采样,从而进一步压缩图像数据中不容易被人眼感知的冗余部分。YUV色彩空间下的色度采样主要有三种采样方式,分别是4:4:4采样、4:4:2采样、4:2:0采样。

\begin{figure}[ht]
    \centering
    \includegraphics[width=1.0\textwidth]{pages/jpeg/chrominance_sample.png}
    \caption{3种色度采样示例}
    \label{Fig.chrominance_sample}
\end{figure}

三种色度采样方式在图\ref{Fig.chrominance_sample}中示例,其中,$J$表示水平抽样参照(概念上区域的宽度,通常为4),$a$表示在J个像素第一行中的色度抽样数目(Cr, Cb),$b$表示在J个像素第二行中的额外色度抽样数目(Cr, Cb)。

4:4:4采样。第一行中,每连续4个点采样4个 CrCb 值,第二行中,每连续4个点采样4个 CrCb 值,即每一个Y对应一个Cr和一个Cb。图像数据中的色度信息完整保留。

4:4:2采样。第一行中,每连续4个点采样2个 CrCb 值,第二行中,每连续4个点采样4个 CrCb 值,即每两个Y共用一对Cr和Cb。图像数据中的仅有50\%的色度信息保留。许多高端数字视频格式和接口都使用这种方案,包括流行的 Apple ProRes422 编码\cite{AppleproResWhiteBook}。

4:2:0采样。第一行中,每连续4个点采样2个 CrCb 值,第二行中,不采样 CrCb ,沿用第一行的 CrCb 值,即每四个Y共用一对Cr和Cb。图像数据中的仅有25\%的色度信息保留。

\begin{figure}[ht]
    \centering
    \begin{subfigure}{0.245\textwidth}
        \includegraphics[width=\linewidth]{pages/jpeg/yuv-y.jpg}
        \caption{YUV-Y通道}
        \label{Fig:RGB2YUV-y}
      \end{subfigure}
      \begin{subfigure}{0.245\textwidth}
        \includegraphics[width=\linewidth]{pages/jpeg/yuv-u.jpg}
        \caption{YUV-U通道}
        \label{Fig:RGB2YUV-u}
      \end{subfigure}
      \begin{subfigure}{0.245\textwidth}
        \includegraphics[width=\linewidth]{pages/jpeg/yuv-v.jpg}
        \caption{YUV-V通道}
        \label{Fig:RGB2YUV-v}
      \end{subfigure}

    \begin{subfigure}{0.245\textwidth}
      \includegraphics[width=\linewidth]{pages/jpeg/yuv420-y.jpg}
      \caption{420YUV-Y通道}
      \label{Fig:RGB2YUV-y}
    \end{subfigure}
    \begin{subfigure}{0.245\textwidth}
      \includegraphics[width=\linewidth]{pages/jpeg/yuv420-u.jpg}
      \caption{420YUV-U通道}
      \label{Fig:RGB2YUV-u}
    \end{subfigure}
    \begin{subfigure}{0.245\textwidth}
      \includegraphics[width=\linewidth]{pages/jpeg/yuv420-v.jpg}
      \caption{420YUV-V通道}
      \label{Fig:RGB2YUV-v}
    \end{subfigure}

    \caption{对YUV色彩空间进行4:2:0色度采样}
    \label{Fig:RGB2YUV}
\end{figure}




\section{DCT变换}
离散余弦变换(Discrete Cosine Transform,DCT)能够将空域的信号转换到频域上,具有良好的去相关性的性能。DCT变换本身是无损的且具有对称性。对原始图像进行离散余弦变换,变换后DCT系数能量主要集中在左上角,其余大部分系数接近于零。将变换后的DCT系数进行门限操作,将小于一定值得系数归零,这就是图像压缩中的量化过程,然后进行逆DCT运算,可以得到压缩后的图像。

对原始图像信号$f(i,j)$进行二维DCT变换得到频域信号$G(u,v)$

\begin{equation}
    \begin{aligned}
        G(u,v) &=c(u)c(v) \sum_{i=0}^{M-1} \sum_{j=0}^{N-1} f(i,j) \cos(\frac{i+0.5}{M}u\pi) \cos(\frac{j+0.5}{N}u\pi) \\
        c(u) &=\left\{\begin{aligned}
            & \sqrt{\frac{1}{N}}, & \quad u=0 \\
            & \sqrt{\frac{2}{N}}, & \quad u\neq 0
        \end{aligned}\right.
        \quad u,v=0,1,2,...,7
    \end{aligned}
    \label{Eq.DCT}
\end{equation}
当 $M=N$ 时,即对方阵进行 DCT 变换时,DCT 变换可以用DCT矩阵进行简化计算, $f$ 的 DCT 变换则是 $G=TfT^T$,其中 DCT 系数矩阵 $T$ 为
\begin{equation}
    T=\frac{2}{\sqrt{N}}
    \begin{bmatrix}
        \frac{1}{\sqrt{2}}      & \frac{1}{\sqrt{2}}        & ...   & \frac{1}{\sqrt{2}} \\
        \cos\frac{\pi}{2N}      & \cos\frac{3\pi}{2N}       & ...   & \cos\frac{(2N-1)\pi}{2N} \\
        ...                     & ...                       &       & ... \\
        \cos\frac{(N-1)\pi}{2N} & \cos\frac{3(N-1)\pi}{2N}  & ...   & \cos\frac{(2N-1)(N-1)\pi}{2N}
    \end{bmatrix}
\end{equation}


当原始图像从 RGB 颜色空间转换到 YUV 颜色空间之后,需要对每一个 $8 \times 8$ 的图像块进行二维DCT变换
\begin{equation}
    \begin{aligned}
        G(u,v) &=c(u)c(v) \sum_{i=0}^{7} \sum_{j=0}^{7} f(i,j) \cos(\frac{i+0.5}{8}u\pi) \cos(\frac{j+0.5}{8}u\pi) \\
        c(u) &=\left\{\begin{aligned}
            & \sqrt{\frac{1}{8}},   & \quad u=0 \\
            & \frac{1}{2},          & \quad u\neq 0
        \end{aligned}\right.
        \quad u,v=0,1,2,...,7
    \end{aligned}
    \label{Eq.DCT8_8}
\end{equation}

由于图像子块是方阵,这时候的 DCT 系数矩阵 $T$ 为
\begin{equation}
    T=\frac{1}{\sqrt{2}}
    \begin{bmatrix}
        \frac{1}{\sqrt{2}}      & \frac{1}{\sqrt{2}}        & ...   & \frac{1}{\sqrt{2}} \\
        \cos\frac{\pi}{16}      & \cos\frac{3\pi}{16}       & ...   & \cos\frac{(16-1)\pi}{16} \\
        ...                     & ...                       &       & ... \\
        \cos\frac{(N-1)\pi}{16} & \cos\frac{3(N-1)\pi}{16}  & ...   & \cos\frac{(16-1)(N-1)\pi}{16}
    \end{bmatrix}
    \label{Eq.DCT-coefficient_matrix}
\end{equation}

在 Matlab 中可以用 \lstinline|T = dctmtx(8)| 查看

\begin{lstlisting}{Matlab}
T =
  0.3536  0.3536  0.3536  0.3536  0.3536  0.3536  0.3536  0.3536
  0.4904  0.4157  0.2778  0.0975 -0.0975 -0.2778 -0.4157 -0.4904
  0.4619  0.1913 -0.1913 -0.4619 -0.4619 -0.1913  0.1913  0.4619
  0.4157 -0.0975 -0.4904 -0.2778  0.2778  0.4904  0.0975 -0.4157
  0.3536 -0.3536 -0.3536  0.3536  0.3536 -0.3536 -0.3536  0.3536
  0.2778 -0.4904  0.0975  0.4157 -0.4157 -0.0975  0.4904 -0.2778
  0.1913 -0.4619  0.4619 -0.1913 -0.1913  0.4619 -0.4619  0.1913
  0.0975 -0.2778  0.4157 -0.4904  0.4904 -0.4157  0.2778 -0.0975
\end{lstlisting}

对图像进行 $8\times 8$ 分块后,对每一个图像子块(patch)矩阵块 P 都进行 DCT 变换 $TPT^T$ 就可以将图像子块从空域变换到频域,从而在频域对图像进行压缩。

\section{量化}
DCT变换之后的结果的图像块有一个特点,低频分量都集中到左上角,高频分量都集中在右下角,这是由于在DCT变换公式\ref{Eq.DCT}中,$\cos\frac{(2x+1)u\pi}{M}=\cos(\frac{u\pi}{8}x+\frac{u\pi}{16})$一项$\frac{u\pi}{8}$表示频率$f$,当$u$越大,频率$f=\frac{u\pi}{8}$越大,也就表示,当$u,v$越大,表示的频率也就越大,当$u=v=0$时,$G(0,0)$也就是直流分量,在变换后的处于左上角第一个数据。此外,DCT变换后的数据,左上角的数值较大,而右下角的数值较小。

变换后的矩阵可以通过一个量化表,将低频分量保留,高频分量去除,以达到压缩图像数据量的目的。这是根据人眼对图像的边缘信息(高频部分)不敏感来进行处理的,以减少人眼不易察觉的高频信息,降低数据冗余。

JPEG的量化计算为
\begin{equation}
    B_{i,j}=round(\frac{G_{i,j}}{Q_{i,j}}),\quad i=0,1,...,7
\end{equation}

其中round函数是取整函数,但考虑到了四舍五入,也就是说
% \begin{equation}
%     round(r)=begin{aligned}
%         2 &\text{when}  1.5 \leq r<2.5 \\
%         1 &\text{when}  0.5 \leq r<1.5 \\
%         0   &\text{when} -0.5 \leq r<0.5 \\
%         -1  &\text{when} -1.5 \leq r<-0.5 \\
%         -2  &\text{when} -2.5 \leq r<-1.5
%     \end{aligned}
% \end{equation}
\begin{equation}
    round(r)=
    \left\{\begin{array}{rcc}
    2   &\text{when}&  1.5\leq r<2.5 \\
    1   &\text{when}&  0.5\leq r<1.5 \\
    0   &\text{when}& -0.5\leq r<0.5 \\
    -1  &\text{when}& -1.5\leq r<-0.5 \\
    -2  &\text{when}& -2.5\leq r<-1.5
    \end{array}\right.
\end{equation}

JPEG使用的标准亮度量化表$Q_Y$和标准色差量化表$Q_C$分别对Y分量和UV分量进行不同程度的量化
\begin{equation}
    Q_Y=\begin{bmatrix}
        16 & 11 & 10 & 16 & 24 & 40 & 51 & 61 \\
        12 & 12 & 14 & 19 & 26 & 58 & 60 & 55 \\
        14 & 13 & 16 & 24 & 40 & 57 & 69 & 56 \\
        14 & 17 & 22 & 29 & 51 & 87 & 80 & 62 \\
        18 & 22 & 37 & 56 & 68 & 109 & 103 & 77 \\
        24 & 35 & 55 & 64 & 81 & 104 & 113 & 92 \\
        49 & 64 & 78 & 87 & 103 & 121 & 120 & 101 \\
        72 & 92 & 95 & 98 & 112 & 100 & 103 & 99
    \end{bmatrix},
    Q_C=\begin{bmatrix}
        17 & 18 & 24 & 47 & 99 & 99 & 99 & 99 \\
        18 & 21 & 26 & 66 & 99 & 99 & 99 & 99 \\
        24 & 26 & 56 & 99 & 99 & 99 & 99 & 99 \\
        47 & 66 & 99 & 99 & 99 & 99 & 99 & 99 \\
        99 & 99 & 99 & 99 & 99 & 99 & 99 & 99 \\
        99 & 99 & 99 & 99 & 99 & 99 & 99 & 99 \\
        99 & 99 & 99 & 99 & 99 & 99 & 99 & 99 \\
        99 & 99 & 99 & 99 & 99 & 99 & 99 & 99
    \end{bmatrix}
\end{equation}
上述的两张量化表只是众多量化表中的两张,我们可以通过是否使用色差量化表$Q_C$还是仅使用亮度量化表$Q_Y$、或者对$Q_Y(Q_C)$量化表进行系数加权得到不同的量化表来对图像进行不同尺度的压缩,以得到数据量更小的图像。

以某一个$8\times 8$的图像块$P$为例,进行量化
\begin{equation}
    P=\begin{bmatrix}
        52 & 55 & 61 & 66 & 70 & 61 & 64 & 73 \\
        63 & 59 & 55 & 90 & 109 & 85 & 69 & 72 \\
        62 & 59 & 68 & 113 & 144 & 104 & 66 & 73 \\
        63 & 58 & 71 & 122 & 154 & 106 & 70 & 69 \\
        67 & 61 & 68 & 104 & 126 & 88 & 68 & 70 \\
        79 & 65 & 60 & 70 & 77 & 68 & 58 & 75 \\
        85 & 71 & 64 & 59 & 55 & 61 & 65 & 83 \\
        87 & 79 & 69 & 68 & 65 & 76 & 78 & 94
        \end{bmatrix}
\end{equation}

由于一个字节的范围是 0\textasciitilde 255 ,为了减小绝对值波动,需要将数值移动到 -128\textasciitilde 128 范围内
\begin{equation}
    P'=\begin{bmatrix}
    -76 & -73 & -67 & -62 & -58 & -67 & -64 & -55 \\
    -65 & -69 & -73 & -38 & -19 & -43 & -59 & -56 \\
    -66 & -69 & -60 & -15 & 16 & -24 & -62 & -55 \\
    -65 & -70 & -57 & -6 & 26 & -22 & -58 & -59 \\
    -61 & -67 & -60 & -24 & -2 & -40 & -60 & -58 \\
    -49 & -63 & -68 & -58 & -51 & -60 & -70 & -53 \\
    -43 & -57 & -64 & -69 & -73 & -67 & -63 & -45 \\
    -41 & -49 & -59 & -60 & -63 & -52 & -50 & -34
    \end{bmatrix}
\end{equation}

接着,根据DCT变换公式,使用DCT系数矩阵(式\ref{Eq.DCT-coefficient_matrix})计算得到变换后的矩阵$G=TPT^T$
\begin{equation}
    G=\begin{bmatrix}
    -415.38 & -30.19 & -61.20 & 27.24 & 56.12 & -20.10 & -2.39 & 0.46 \\
    4.47 & -21.86 & -60.76 & 10.25 & 13.15 & -7.09 & -8.54 & 4.88 \\
    -46.83 & 7.37 & 77.13 & -24.56 & -28.91 & 9.93 & 5.42 & -5.65 \\
    -48.53 & 12.07 & 34.10 & -14.76 & -10.24 & 6.30 & 1.83 & 1.95 \\
    12.12 & -6.55 & -13.20 & -3.95 & -1.87 & 1.75 & -2.79 & 3.14 \\
    -7.73 & 2.91 & 2.38 & -5.94 & -2.38 & 0.94 & 4.30 & 1.85 \\
    -1.03 & 0.18 & 0.42 & -2.42 & -0.88 & -3.02 & 4.12 & -0.66 \\
    -0.17 & 0.14 & -1.07 & -4.19 & -1.17 & -0.10 & 0.50 & 1.68
    \end{bmatrix}
\end{equation}


根据亮度量化表,得到量化系数矩阵(Quantization matrices)$Q$
\begin{equation}
    Q=\begin{bmatrix}
    16 & 11 & 10 & 16 & 24 & 40 & 51 & 61 \\
    12 & 12 & 14 & 19 & 26 & 58 & 60 & 55 \\
    14 & 13 & 16 & 24 & 40 & 57 & 69 & 56 \\
    14 & 17 & 22 & 29 & 51 & 87 & 80 & 62 \\
    18 & 22 & 37 & 56 & 68 & 109 & 103 & 77 \\
    24 & 35 & 55 & 64 & 81 & 104 & 113 & 92 \\
    49 & 64 & 78 & 87 & 103 & 121 & 120 & 101 \\
    72 & 92 & 95 & 98 & 112 & 100 & 103 & 99
    \end{bmatrix}
\end{equation}

使用量化矩阵就可以得到量化结果
\begin{equation}
    B=\begin{bmatrix}
    -26 & -3 & -6 & 2 & 2 & -1 & 0 & 0 \\
    0 & -2 & -4 & 1 & 1 & 0 & 0 & 0 \\
    -3 & 1 & 5 & -1 & -1 & 0 & 0 & 0 \\
    -3 & 1 & 2 & -1 & 0 & 0 & 0 & 0 \\
    1 & 0 & 0 & 0 & 0 & 0 & 0 & 0 \\
    0 & 0 & 0 & 0 & 0 & 0 & 0 & 0 \\
    0 & 0 & 0 & 0 & 0 & 0 & 0 & 0 \\
    0 & 0 & 0 & 0 & 0 & 0 & 0 & 0
    \end{bmatrix}
\end{equation}


量化后的数值都较小,并且存在较多的0数据,这些0数据都可以看成是冗余数据,可以用游程编码大大减小数据量。


\section{游程编码}
游程编码又称“运行长度编码”或“行程编码”,是一种统计编码。该编码属于无损压缩编码,是栅格数据压缩的重要编码方法,广泛应用于二值图像的压缩。它的基本思路是:对于一幅栅格图像,常常有行(或列)方向上相邻的若干点具有相同的属性代码,因而可采取某种方法压缩那些重复的记录内容。其编码方案是,只在各行(或列)数据的代码发生变化时依次记录该代码以及相同代码重复的个数,从而实现数据的压缩。游程编码的优点在于压缩效率较高,易于检索、叠加合并等操作,运算简单,适用于机器存贮容量小,数据需大量压缩,而又要避免复杂的编码解码运算增加处理和操作时间的情况。

例如:5555557777733322221111111

游程编码为:(5,6)(7,5)(3,3)(2,4)(l,7)。可见,用一个(值,串长)代替具有相同值的连续符号,构成一段连续的“行程”,大多数情况下,该符号长度少于原始数据的长度,故最终编码结果的位数远远少于原始字符串的位数。

在对图像数据进行编码时,沿一定方向排列的具有相同灰度值的像素可看成是连续符号,用字串代替这些连续符号,可大幅度减少数据量。游程编码是连续精确的编码,在传输过程中,如果其中一位符号发生错误,即可影响整个编码序列,使游程编码无法还原回原始数据。游程编码所能获得的压缩比有多大,主要取决于图像本身的特点。如果图像中具有相同颜色的图像块越大,图像块数目越少,获得的压缩比就越高。反之,压缩比就越小。

直接对自然图片进行游程编码压缩显然是不可行的,因为自然图片像素点错综复杂,同色像素连续性差,连续像素亮度值完全相等的概率低,如果应用游程编码方法来编码就适得其反,图像体积不但没减少,反而加倍。但在本例中,经zigzag变换后的量化系数向量中存在大量的连0,使用游程编码显然可以大幅压缩数据量。

\section{熵编码}
\subsection{霍夫曼编码}
霍夫曼编码是一种用于无损数据压缩的熵编码(权编码)算法,其使用变长编码表对数据进行编码。其中变长编码表是通过一种评估来源符号出现概率的方法得到的,出现概率高的字母使用较短的编码,反之出现概率低的则使用较长的编码,这便使编码之后的字符串的平均长度、期望值降低,从而达到无损压缩数据的目的。

例如,在英文中,e的出现概率最高,而z的出现概率则最低。当利用霍夫曼编码对一篇英文进行压缩时,e极有可能用一个比特来表示,而z则可能花去25个比特(不是26)。用普通的表示方法时,每个英文字母均占用一个字节,即8个比特。二者相比,e使用了一般编码的1/8的长度,z则使用了3倍多。倘若我们能实现对于英文中各个字母出现概率的较准确的估算,就可以大幅度提高无损压缩的比例。

霍夫曼编码主要分为两步:
\begin{enumerate}
    \item 根据每个字符出现的概率构建一颗最优二叉树;
    \item 再根据二叉树对字符进行编码。
\end{enumerate}

构建霍夫曼树的过程如下:
\begin{enumerate}
    \item 统计文本中字符出现的次数;
    \item 将字符按照频数升序排序;
    \item 将频数最小的两个叶子结点结合成树,看作一个整体,整体的频数是叶子结点频数和;
    \item 把这个树看作整体和其他的一起也进行升序排序;
    \item 重复上述过程直到生成整棵树,最后产生的树状图就是霍夫曼树。
\end{enumerate}
在根据字符出现的频次构建好霍夫曼树后,还需要对每个字符进行编码,编码过程如下:
\begin{enumerate}
    \item 给霍夫曼树的所有左链接'0'与右链接'1';
    \item 从树根至树叶依序记录所有字母的编码。
\end{enumerate}
例如:对于一个包含了6种不同字符的字符串,其符号出现的频率如下:
\begin{table}\centering
  \caption{不同符号的频数}
    \begin{tabular}[]{|c|c|c|c|c|c|c|}
      \hline
        符号 & F & O & E & R & G & T \\ \hline
        频数 & 2 & 3 & 5 & 4 & 4 & 7 \\ \hline
    \end{tabular}
\end{table}
由此根据上述步骤能够构建出最优霍夫曼树,以及对应的编码,如图所示:
\begin{figure}[h]\centering
    \includegraphics[width = 0.4\linewidth]{pages/jpeg/HuffmanCoding.jpg}
\end{figure}
